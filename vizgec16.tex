\documentclass{sig-alternate}

% *** SPECIALIZED LIST PACKAGES ***
%
\usepackage{algorithmic}

\usepackage{array}

% *** PDF, URL AND HYPERLINK PACKAGES ***
%
\usepackage{url}

\begin{document}

% Copyright
\setcopyright{acmcopyright}
%\setcopyright{acmlicensed}
%\setcopyright{rightsretained}
%\setcopyright{usgov}
%\setcopyright{usgovmixed}
%\setcopyright{cagov}
%\setcopyright{cagovmixed}

%Conference
\conferenceinfo{GECCO '16}{Denver, CO, USA}
% --- End of Author Metadata ---

\title{Visualizing for success: making the user more efficient in
  interactive evolutionary algorithms}
%
% You need the command \numberofauthors to handle the 'placement
% and alignment' of the authors beneath the title.
%
% For aesthetic reasons, we recommend 'three authors at a time'
% i.e. three 'name/affiliation blocks' be placed beneath the title.
%
% NOTE: You are NOT restricted in how many 'rows' of
% "name/affiliations" may appear. We just ask that you restrict
% the number of 'columns' to three.
%
% Because of the available 'opening page real-estate'
% we ask you to refrain from putting more than six authors
% (two rows with three columns) beneath the article title.
% More than six makes the first-page appear very cluttered indeed.
%
% Use the \alignauthor commands to handle the names
% and affiliations for an 'aesthetic maximum' of six authors.
% Add names, affiliations, addresses for
% the seventh etc. author(s) as the argument for the
% \additionalauthors command.
% These 'additional authors' will be output/set for you
% without further effort on your part as the last section in
% the body of your article BEFORE References or any Appendices.

\numberofauthors{2} 
\author{
\alignauthor
Juan-J.~Merelo, \titlenote{Corresponding author}\\
\affaddr{Dept. of Computer Architecture and Technology and CITIC}\\
\affaddr{University of Granada, Granada, Spain} \\
\email{jmerelo@ugr.es}
\alignauthor
Mario Garc\'ia-Valdez, 
\affaddr{Dept. of Graduate Studies}\\
\affaddr{ Instituto Tecnol�gico de Tijuana, Tijuana, M\'exico}\\
}

\maketitle

\begin{abstract}

Using volunteer's browsers as a computing resource presents several
advantages, but it remains a challenge to fully harness the browser's
capabilities and to model the user's behavior so that those
capabilities can be leveraged optimally. 

\end{abstract}

\keywords{Volunteer computing, distributed computing, cloud computing,
visualization}


%---------------------------------------------------------------
\section{Introduction}




%---------------------------------------------------------------
\section{State of the art}
\label{sec:soa}



\section{Methodology and experimental results}
\label{sec:exp1}



%---------------------------------------------------------------
\section{Conclusions and future work}
\label{sec:conclusion}


%---------------------------------------------------------------
\section*{Acknowledgment}
 
This work has been supported in part by
TIN2014-56494-C4-3-P (Spanish Ministry of Economy and Competitivity), PROY-PP2015-06 (Plan Propio 2015 UGR). Additional support was received by
Projects 5622.15-P (ITM) and  PROINNOVA 2015: 220590 (CONACYT).

\bibliographystyle{abbrv}
\bibliography{geneura,volunteer,javascript,ror-js,GA-general}

\end{document}
%%% Local Variables:
%%% ispell-local-dictionary: "english"
%%% End:
